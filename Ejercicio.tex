\documentclass{article}

\usepackage[utf8]{inputenc}
\usepackage[spanish]{babel}
\usepackage{amsmath}

\begin{document}
Escribe la solución de la ecuación en diferencias
$$x_{n+2}-4x_n=0$$
con condiciones iniciales
\begin{align*}
  x_0&=1\\
  x_1&=-1
\end{align*}

\begin{center}
 Procedimiento:
\end{center}
Evaluamos n con 0, 1 y 2 para obtener otros de los valores iniciales\\
$$n=0, \ x_2=4x_0=4$$
$$n=1, \ x_3=4x_1=-4$$
$$n=2, \ x_4=4x_2=16$$
Después de evaluar los primeros valores para $n$, con la ecuación
$$x_{n+2}-4x_n=0,$$
tenemos lo siguiente
$$r^2-4=0$$
Despejamos r, para obtener $r_1$ y $r_2$\\
$$=> r_1=2, r_2=-2$$
Teniendo los valores para $r_1$ y $r_2$ sustituimos en la formula
$x_n=\alpha_1(r_1)^n+\alpha_2(r_2)^n$
y queda de la siguiente manera
$x_n=\alpha_1(2)^n+\alpha_2(-2)^n$
Ahora evaluando los valores iniciales en la ecuación anterior haremos un sistema de ecuaciones para obtener los valores de $\alpha_1 y \alpha_2$
\begin{align*}
  => x_0&=\alpha_1(2)^0+\alpha_2(-2)^0\\
        &=\alpha_1+\alpha_2\\
        &=1\\
  => x_1&=\alpha_1(2)^1+\alpha_2(-2)^1\\
        &=(2)\alpha_1+(-2)\alpha_2\\
        &=-1
\end{align*}
Con esto despejamos en cada una de las $x$ a $\alpha_1$ entonces se tiene lo siguiente
\begin{align*}
  \alpha_1&=1-\alpha_2\\
  \alpha_1&=\frac{-1+2\alpha_2}{2}
\end{align*}


\end{document}
